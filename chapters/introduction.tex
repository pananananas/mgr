\Chapter{Introduction TODO}\label{chapter:introduction}
W pracy formułuje się cele o charakterze badawczym wymagające doboru i zastosowania metod badawczych, wykorzystując wiedzę teoretyczną oraz naukową. Wskazane jest przedstawienie, co nowego jest zaproponowane w pracy oraz podanie ograniczeń i słabych/mocnych stron opracowanego rozwiązania (jeżeli dotyczy). Rozdział wprowadzający powinien służyć czytelnikowi do zrozumienia celu pracy.

\section{Problem Statement}
W tej sekcji student powinien przedstawić bliżej problem, którym chce się zmierzyć. Jasno zdefiniuj problem badawczy. Podaj swoje cele, zadania i pytania badawcze. Wyjaśnij znaczenie badania. Określ ograniczenia badań.


\section{Thesis Objectives}
W tej sekcji powinny zostać przedstawione konkretne działania, które określają pracę studenta w celu rozwiązania problemu.


\section{Thesis Outline}
Zarysuj strukturę swojej pracy dyplomowej. Ogólnie przedstawienie pracy. Przykładowo: ,,Praca dzieli się na $7$ rozdziałów (\dots)''. Rozdział \ref{chapter:politechnika} dotyczy (\dots). Temat został rozwinięty~w~\ref{chapter:podrozdzial}.